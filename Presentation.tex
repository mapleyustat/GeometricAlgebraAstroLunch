%% NOTE: This file is just included from PresentationWithNotes.tex or
%% from PresentationSlides.tex.
\documentclass[12pt,xcolor={dvipsnames}]{beamer}

\input{Externals/UsePackages} %
% \pagecolor{white}\usepackage{pdfpages} % For including other PDFs
\input{Externals/Macros} %
\input{Externals/Cornell} %
%\usepackage{multimedia} %
%\usepackage[setpagesize=false]{hyperref} %
%\usepackage{pgfpages} %
\usepackage[overlay]{textpos} %
\usepackage[T1]{fontenc}
\pdfmapfile{+Clifford.map}
\DeclareFontFamily{U}{Clifford}{\hyphenchar\font=-1}
\DeclareFontShape{U}{Clifford}{m}{n}{<->Clifford}{}
\makeatletter
\def\@Cl#1{%
  \mbox{{\fontsize{#1}{#1}\fontfamily{Clifford}%
    \fontencoding{U}\fontseries{m}\fontshape{n}\selectfont%
    \char "43}}}
\def\Cl{\mathchoice
  {\@Cl{\tf@size}}
  {\@Cl{\tf@size}}
  {\@Cl{\sf@size}}
  {\@Cl{\ssf@size}}}
\makeatother


\setlength{\TPHorizModule}{1cm}
\setlength{\TPVertModule}{1cm}

\renewcommand{\vec}{\bm}

\setbeamercolor{normal text}{fg=CornellDarkGray,%
  bg=CornellLightestGray}
\titlegraphic{\includegraphics[width=75pt, height=75pt]%
  {Externals/CULogoFineGray}}


%% For tikz/pgf plots
\usepackage{tikz} %
% \usepackage{tikzscale} %
\usepackage{pgfplots} %
\usepackage{pgfplotstable} %
\pgfplotsset{width=330pt} %
\pgfplotsset{compat=1.7} %
\usetikzlibrary{pgfplots.groupplots} %
%\usetikzlibrary{decorations.shapes} %
%\usetikzlibrary{decorations.markings} %
%\usetikzlibrary{decorations.pathreplacing, calc} %
\pgfplotsset{axis line style={line width=1pt, black}} %
\pgfplotsset{tick style={line width=1pt, black}} %
\pgfplotsset{minor tick style={black}} %
\newcommand{\PlotData}{
  \addplot[color=Red, densely dotted, very thick]
  table[x index=0, y index=1] {\loadedtable};
  \addplot[color=Blue, dotted, very thick]
  table[x index=0, y index=2] {\loadedtable};
  \addplot[smooth, color=Green, very thick]
  table[x index=0, y index=3] {\loadedtable};
  \addplot[color=Blue]
  table[x index=0, y index=4] {\loadedtable};
  \addplot[color=Red]
  table[x index=0, y index=5] {\loadedtable};
}
%% Automatically write figures as PDFs for import next time
\usetikzlibrary{external}
\tikzexternalize
\tikzsetfigurename{Presentation_} %
%\tikzset{external/force remake} % To force reprocessing of plots


\renewcommand{\vector}[1]{\bm{#1}}
\newcommand{\vectordot}[1]{\bm{{\dot{#1}}}}
\newcommand{\unitvec}[1]{\vector{{\hat{#1}}}}
\newcommand{\unitvecdot}[1]{\vector{{\dot{\hat{#1}}}}}

\newcommand{\quat}[1]{\ensuremath{\mathbf{\MakeUppercase{#1}}}}
\newcommand{\quatcomponent}[2]{\ensuremath{\MakeLowercase{#1}_{#2}}}
\newcommand{\quatcomponentco}[2]%
        {\ensuremath{\co{\MakeLowercase{#1}}_{#2}}}
\newcommand{\quatpart}[2]{\ensuremath{\MakeUppercase{#1}_{#2}}}
\newcommand{\quatpartco}[2]{\ensuremath{\co{\MakeUppercase{#1}}_{#2}}}
\newcommand{\quatabs}[1]{\ensuremath{\abs{\quat{#1}}}}
\newcommand{\quatvec}[1]{\ensuremath{\vector{\lowercase{#1}}}}
\newcommand{\quatvecmag}[1]{\ensuremath{\lowercase{#1}}}
\newcommand{\quatunitvec}[1]{\ensuremath{\unitvec{\lowercase{#1}}}}
\newcommand{\quatlog}[1]{\ensuremath{\bm{\mathfrak{\lowercase{#1}}}}}
\newcommand{\quatlogmag}[1]{\ensuremath{\mathfrak{\lowercase{#1}}}}
\DeclareMathOperator{\Angle}{ang}
\newcommand{\quatangle}[1]{\ensuremath{\Angle{\quat{#1}}}}

\DeclareMathOperator*{\DirectSum}{\oplus}
\DeclareMathOperator*{\DirectProduct}{\otimes}
\renewcommand{\G}{\mathcal{G}}

\graphicspath{{Plots/}} %

\title{The Grand Unified Theory {\fontsize{0.15cm}{1em}\selectfont of
    geometry}}
\subtitle{\small Geometric Algebra}

\institute[Cornell Astrophysics Lunch]{Cornell Astrophysics Lunch}

\date{October 29, 2014}

\author[Mike Boyle] {Mike Boyle}

\subject{}

\begin{document}

\begin{frame}[plain]
  \titlepage

  \note{I'll start off by admitting that I was primarily a math major
    in undergrad at U. Chicago, and I absolutely loved it. Now, if you
    don't know the math department at Chicago, I'll tell you that that
    basically means I absolutely love useless mathematical formalism
    and obscure proofs. And some of what I'll talk about has buzzwords
    associated with it that might tune people out.  Please don't let
    it.

    Because I'm a physicist now, and useful math is my bread and
    butter. And the message I really hope to get across is that GA is
    \emph{extraordinarily} useful, concrete, and easy. In fact, it is
    so useful that I can honestly say that I think it's basically all
    the math a physicist needs to know. I sincerely think that it can
    be called the grand unified theory of geometry --- or at least the
    parts of geometry that physicists care about. It ties together
    every technique I've ever seen in mathematical physics; it does it
    very simply; and it's easier to solve problems with it.
    ridiculously simple.

    Three basic points: It is easy to understand. It takes every
    separate strand of mathematics physicists use, and ties them
    together into one tidy bundle. And to top it off, it's easier to
    solve problems with GA than with standard traditional methods.}
\end{frame}

%%%%%%%%%%%%%%%%%%%%%%%%%%%%%%%%%%%%%%%%%%%%%%%%%%%%%%%%%%%%%%%%%%%%%%
%%%%%%%%%%%%%%%%%%%%%%%%%%%%%%%%%%%%%%%%%%%%%%%%%%%%%%%%%%%%%%%%%%%%%%

%%%%%%%%%%%%%%%%%%%%%%%%%%%%%%%%%%%%%%%%%%%%%%%%%%%%%%%%%%%%%%%%%%%%%%
%%%%%%%%%%%%%%%%%%%%%%%%%%%%%%%%%%%%%%%%%%%%%%%%%%%%%%%%%%%%%%%%%%%%%%

\begin{frame}
  \frametitle{Geometric algebra (GA)}
  \begin{enumerate}
   \item GA is easy to understand (and teach)
   \item It unifies and extends familiar math
   \item It makes problems easier
  \end{enumerate}
\end{frame}

\begin{frame}
  \frametitle{Outline}
  \begin{itemize}
   \item Geometric Algebra (GA)
   \item GA in two dimensions
   \item GA in more dimensions
   \item Transformations with GA
   \item Keplerian orbits
   \item Geometric calculus
   \item Dirac equation
   \item E\&M
   \item Riemann tensor for Kerr
  \end{itemize}
\end{frame}

%%%%%%%%%%%%%%%%%%%%%%%%%%%%%%%%%%%%%%%%%%%%%%%%%%%%%%%%%%%%%%%%%%%%%%
%%%%%%%%%%%%%%%%%%%%%%%%%%%%%%%%%%%%%%%%%%%%%%%%%%%%%%%%%%%%%%%%%%%%%%

% \part{Introduction}
% \frame{\partpage}

\begin{frame}
  \frametitle{Free\textsuperscript{1} algebra}
  \begin{textblock}{4}(0.,7.5) \scriptsize{\textsuperscript{1}unital,
      associative}
  \end{textblock}

  Letters: \vspace{-0.1in}
  \begin{center}
    Scalars $\alpha, \beta, \gamma, \delta, \ldots \in \mathbb{R}$

    Vectors $\vec{v}, \vec{w}, \vec{m}, \vec{n}, \ldots \in
    \mathbb{R}^{p,q}$
  \end{center}
  \pause \vspace{-0.05in}

  Words: \vspace{-0.2in}
  \begin{gather*}
    \vec{v}
    \\
    \alpha\, \vec{w}
    \\
    \alpha\, \vec{v}\, \vec{w}
    \\
    (\alpha\, \vec{v}\, \vec{w})\, (\beta\, \vec{m}\, \vec{n}) =
    \alpha\, \beta\, \vec{v}\, \vec{w}\, \vec{m}\, \vec{n}
  \end{gather*}
  \pause \vspace{-0.2in}

  Sentences: \vspace{-0.15in}
  \begin{gather*}
    \vec{v} + \gamma\, \vec{w}\, \vec{m}\, \vec{n} + \alpha\, \beta\,
    \vec{v}\, \vec{w}\, \vec{m}\, \vec{n}
    \\
    \gamma\, \vec{v}(1 + \delta\, \vec{m}\, \vec{n}) = \gamma\,
    \delta\, \vec{v}\, \vec{m}\, \vec{n} + \gamma\, \vec{v}
  \end{gather*}

  \note{But now the question is: what do we do with this?}
\end{frame}

\begin{frame}
  \frametitle{Free algebra is \emph{too} free!}
  Allow manipulation
  \begin{equation*}
    \vec{v}\, \vec{v} \leftrightarrow \vec{v} \cdot \vec{v}
  \end{equation*}
  \pause

  We get inverses!
  \begin{equation*}
    1
    = \frac{\vec{v}\cdot \vec{v}} {\vec{v}\cdot \vec{v}}
    \pause
    = \frac{\vec{v}\, \vec{v}} {\vec{v}\cdot \vec{v}}
    \pause
    = \vec{v}\, \left(\frac{\vec{v}} {\vec{v}\cdot\vec{v}}\right)
  \end{equation*}
  \pause
  \begin{equation*}
    \vec{v}^{-1} = \frac{\vec{v}} {\vec{v} \cdot \vec{v}}
  \end{equation*}

  \note{We want some way to manipulate these expressions further. One
    thing that would be really nice is if we had inverses. But the
    only thing around with any inverses is our field, $\mathbb{R}$, so
    we need to simplify some of our expressions into scalars. In
    particular, it makes sense to impose $\vec{v}\, \vec{v} = \vec{v}
    \cdot \vec{v} \in \mathbb{R}$.

    Note that this doesn't imply $\vec{v}\, \vec{w} = \vec{v} \cdot
    \vec{w}$; our substitution only works for a vector times itself.}
\end{frame}

\begin{frame}
  \frametitle{General Clifford algebra}

  \begin{equation*}
    \Cl(\mathbb{V}, N) = \DirectSum_{g=0}^{\infty} \otimes^{g}\,
    \mathbb{V}\ /\ \{ \vec{v}\otimes \vec{v} - N(\vec{v}) \}
  \end{equation*}
\end{frame}

\begin{frame}
  \frametitle{Two dimensions \phantom{:$\G(\mathbb{R}^{2})$}}
  \begin{equation*}
    \vec{v} = \vec{x} + \vec{y}
  \end{equation*}
  \pause
  \begin{equation*}
    \vec{v}\, \vec{v} = \vec{v} \cdot \vec{v} = 2
  \end{equation*}
  \pause \vspace{-0.125in}
  \begin{align*}
    \vec{v}\, \vec{v} &= (\vec{x}+\vec{y})\, (\vec{x}+\vec{y}) \\
    &= \vec{x}\, \vec{x} + \vec{y}\, \vec{y} + \vec{x}\, \vec{y} +
    \vec{y}\, \vec{x} \\
    &= 2 + (\vec{x}\, \vec{y} + \vec{y}\, \vec{x})
  \end{align*}
  \vspace{-0.175in}

  \pause
  \begin{equation*}
    \vec{x}\, \vec{y} = -\vec{y}\, \vec{x}
  \end{equation*}
\end{frame}

\begin{frame}
  \frametitle{Two dimensions: $\mathbb{C}$
    \phantom{:$\G(\mathbb{R}^{2})$}}
  \begin{equation*}
    \vec{x}\, \vec{y} = -\vec{y}\, \vec{x}
  \end{equation*}

  \begin{equation*}
    (\vec{x}\, \vec{y})^{2} = \vec{x}\, \vec{y}\, \vec{x}\, \vec{y} =
    -\vec{y}\, \vec{x}\, \vec{x}\, \vec{y} = - \vec{y}\, \vec{y} = -1
  \end{equation*}

  \pause \vspace{0.1in}
  \begin{center}
    The quantity $\vec{x}\, \vec{y}$ \emph{is} the unit imaginary $i$
    (in 2-d)

    The quantity $\vec{y}\, \vec{x}$ \emph{is} the conjugate $\bar{i}$
    (in 2-d)
  \end{center}

  \pause
  \begin{align*}
    \vec{v}\, \vec{w} &= (v_{x}\vec{x} + v_{y}\vec{y})\, (w_{x}\vec{x}
    + w_{y}\vec{y}) \\
    &= (v_{x}\, w_{x} + v_{y}\, w_{y}) + (v_{x}\, w_{y} - v_{y}\,
    w_{x}) \vec{x}\, \vec{y} \\
    \only<4->{&= \vec{v} \cdot \vec{w} + \vec{v} \wedge \vec{w}}
  \end{align*}

  \note{Product of two vectors is a spinor. You get one term that's
    the dot product between the vectors, and one that's the area of
    the parallelogram.}
\end{frame}

\begin{frame}
  \frametitle{Two dimensions: $\G(\mathbb{R}^{2})$}
  \begin{equation*}
    1
    \qquad
    \begin{matrix}
      \vec{x} \\ \vec{y}
    \end{matrix}
    \qquad
    \vec{x}\, \vec{y}
  \end{equation*}

  \note{Vectors get re-mapped into spinors. Accident of dimension.
    Easy to generalize dimensions}
\end{frame}

\begin{frame}
  \frametitle{Three dimensions: $\G(\mathbb{R}^{3})$}
  \begin{equation*}
    1
    \qquad
    \begin{matrix}
      \vec{x} \\ \vec{y} \\ \vec{z}
    \end{matrix}
    \qquad
    \begin{matrix}
      \vec{x}\,\vec{y} \\ \vec{y}\, \vec{z} \\ \vec{z}\, \vec{x}
    \end{matrix}
    \qquad
    \vec{x}\, \vec{y}\, \vec{z}
  \end{equation*}
  \pause

  \begin{equation*}
    1
    \qquad
    \begin{matrix}
      \sigma_{x} \\ \sigma_{y} \\ \sigma_{z}
    \end{matrix}
    \qquad
    \begin{matrix}
      \sigma_{x}\,\sigma_{y} \\ \sigma_{y}\, \sigma_{z} \\
      \sigma_{z}\, \sigma_{x}
    \end{matrix}
    \qquad
    \sigma_{x}\, \sigma_{y}\, \sigma_{z}
  \end{equation*}
  \note{Quaternions are remapped spinors. Pauli matrices are folded
    copies of the whole algebra.}
\end{frame}

\begin{frame}
  \frametitle{Spacetime: $\G(\mathbb{R}^{3,1})$}
  \begin{equation*}
    1
    \qquad
    \begin{matrix}
      \vec{x} \\ \vec{y} \\ \vec{z} \\ \vec{t}
    \end{matrix}
    \qquad
    \begin{matrix}
      \vec{x}\,\vec{y} \\
      \vec{y}\, \vec{z} \\
      \vec{z}\, \vec{x} \\
      \vec{x}\,\vec{t} \\
      \vec{y}\,\vec{t} \\
      \vec{z}\,\vec{t}
    \end{matrix}
    \qquad
    \begin{matrix}
      \vec{x}\, \vec{y}\, \vec{z} \\
      \vec{y}\, \vec{z}\, \vec{t} \\
      \vec{z}\, \vec{t}\, \vec{x} \\
      \vec{t}\, \vec{x}\, \vec{y}
    \end{matrix}
    \qquad
    \vec{x}\, \vec{y}\, \vec{z}\, \vec{t}
  \end{equation*}
  \pause

  \begin{equation*}
    1
    \qquad
    \begin{matrix}
      \gamma_{x} \\ \gamma_{y} \\ \gamma_{z} \\ \gamma_{t}
    \end{matrix}
    \qquad
    \begin{matrix}
      \gamma_{x}\,\gamma_{y} \\
      \gamma_{y}\, \gamma_{z} \\
      \gamma_{z}\, \gamma_{x} \\
      \gamma_{x}\,\gamma_{t} \\
      \gamma_{y}\,\gamma_{t} \\
      \gamma_{z}\,\gamma_{t}
    \end{matrix}
    \qquad
    \begin{matrix}
      \gamma_{x}\, \gamma_{y}\, \gamma_{z} \\
      \gamma_{y}\, \gamma_{z}\, \gamma_{t} \\
      \gamma_{z}\, \gamma_{t}\, \gamma_{x} \\
      \gamma_{t}\, \gamma_{x}\, \gamma_{y}
    \end{matrix}
    \qquad
    \gamma_{x}\, \gamma_{y}\, \gamma_{z}\, \gamma_{t}
  \end{equation*}
\end{frame}

%%%%%%%%%%%%%%%%%%%%%%%%%%%%%%%%%%%%%%%%%%%%%%%%%%%%%%%%%%%%%%%%%%%%%%
%%%%%%%%%%%%%%%%%%%%%%%%%%%%%%%%%%%%%%%%%%%%%%%%%%%%%%%%%%%%%%%%%%%%%%

\begin{frame}
  \frametitle{Reflection}
  \begin{align*}
    \vec{n}\, \vec{v}\, \vec{n}^{-1} %
    &= \vec{n}\, (\vec{v}_{\parallel} + \vec{v}_{\perp})\,
    \vec{n}^{-1} \\ %
    &= (\vec{v}_{\parallel} - \vec{v}_{\perp})\, \vec{n}\,
    \vec{n}^{-1} \\ %
    &= \vec{v}_{\parallel} - \vec{v}_{\perp}
  \end{align*}

  \note{Arbitrary dimension and signature!}
\end{frame}

\begin{frame}
  \frametitle{Orthogonal transformations}
  \emph{Any} orthogonal transformation in $p+q$ dimensions can be
  written as the composition of at most $p+q$ reflections \pause
  \vspace{0.2in}

  Any \emph{special} orthogonal transf. in $p+q$ dimensions can be
  written as the composition of at most $\lfloor \frac{p+q}{2}
  \rfloor$ \emph{pairs of} reflections \pause \vspace{0.2in}

  Product of vectors gives ``Pin'' group, pinors\\
  $\text{Pin}(p,q) \rightarrow \text{O}(p,q)$\vspace{0.2in}

  Product of \emph{pairs of} vectors gives ``Spin'' group, spinors\\
  $\text{Spin}(p,q) \rightarrow \text{SO}(p,q)$
\end{frame}

\begin{frame}
  \frametitle{General linear transformations}
  $m \in GL(p,q)$:
  \begin{equation*}
    \vec{v} \mapsto M\, \vec{v}\, M^{-1}
  \end{equation*}

  \pause \vspace{0.2in}

  \emph{All} finite-dimensional Lie algebras:
  \begin{equation*}
    \left[\vec{a}\, \vec{b}, \vec{c}\, \vec{d} \right]
  \end{equation*}
\end{frame}

\begin{frame}
  \frametitle{Rotations \only<3->{(and boosts)}}
  \begin{gather*}
    \vec{v}\ \mapsto\ \vec{m}\, (\vec{n}\, \vec{v}\, \vec{n}^{-1})\,
    \vec{m}^{-1}\ =\ (\vec{m}\, \vec{n})\, \vec{v}\, (\vec{m}\,
    \vec{n})^{-1}
    \\[5pt]
    \vec{m}\, \vec{n} = \cos\theta + \frac{\vec{m} \wedge
      \vec{n}}{\lvert \vec{m} \wedge \vec{n} \rvert}\, \sin\theta
  \end{gather*}
  \pause \vspace{-0.1in}

  \begin{gather*}
    e^{\theta\, \vec{a}\, \vec{b}} = \cos\theta + \vec{a}\, \vec{b}\,
    \sin\theta
    \\[5pt]
    \vec{v} \mapsto e^{\theta\, \vec{a}\, \vec{b}/2}\, \vec{v}\,
    e^{-\theta\, \vec{a}\, \vec{b}/2}
  \end{gather*}
  \pause \vspace{-0.1in}

  % \begin{align*}
  %   e^{\theta\, \vec{a}\, \vec{b}/2}\, \vec{v}\, e^{-\theta\,
  %   \vec{a}\, \vec{b}/2} %
  %   &= e^{\theta\, \vec{a}\, \vec{b}/2}\, (\vec{v}_{\parallel} +
  %   \vec{v}_{\perp})\, e^{-\theta\, \vec{a}\, \vec{b}/2} \\ %
  %   &= e^{\theta\, \vec{a}\, \vec{b}}\, \vec{v}_{\parallel} +
  %   \vec{v}_{\perp} \\ %
  % \end{align*}
  % \pause

  \begin{gather*}
    e^{\alpha\, \vec{a}\, \vec{t}} = \cosh\alpha + \vec{a}\, \vec{t}\,
    \sinh\alpha
    \\[5pt]
    \vec{v} \mapsto e^{\alpha\, \vec{a}\, \vec{t}/2}\, \vec{v}\,
    e^{-\alpha\, \vec{a}\, \vec{t}/2}
  \end{gather*}

  \note{I got into GA originally because this other group had written
    this paper about finding the symmetry axis of a gravitational
    waveform. But they came up with this rotation that just took the
    $z$ axis directly onto the symmetry axis, and ignored everything
    else. The problem with this is that ``everything else'' is the
    phase of the gravitational wave, which you may know is a very
    important quantity. In their system, the phase was terrible: it
    had discontinuities; it could reverse direction; it was just
    awful. The problem was that they were only keeping track of that
    one vector; they had their eyes on one thing that was being
    transformed, and that's not enough to tell you what's going on
    around that vector. So Rob Owen, Harald Pfeiffer, and I pointed
    out that you need to keep track of the thing that's doing the
    transformation.}

  \note{We deal with equations describing the change in things that
    get transformed. So the question is: do we deal with the thing
    itself that gets transformed, or the thing doing the transforming?
    And I think a pretty good rule of thumb for answering this
    question is: If there's only one thing that can get transformed,
    then by all means just look at that; if there are more, then you
    would probably benefit from looking at the thing doing the
    transforming.}
\end{frame}

\begin{frame}
  \frametitle{Keplerian orbits}
  \begin{equation*}
    \vec{r} = \quat{U}\, \hat{\vec{r}}_0\, \bar{\quat{U}} =
    \quat{U}^{2}\, \hat{\vec{r}}_0
  \end{equation*}
  \pause
  \begin{equation*}
    \dot{\vec{r}} = 2\, \dot{\quat{U}}\, \quat{U}\, \hat{\vec{r}}_0
    \quad \text{and} \quad r = \quat{U}\, \bar{\quat{U}}
  \end{equation*}
  \pause
  \begin{equation*}
    r\, \dot{\quat{U}} = \frac{1}{2} \dot{\vec{r}}\, \quat{U}\,
    \hat{\vec{r}}_0 \pause \qquad \rightarrow \qquad \frac{d}{ds}
    \quat{U} = \frac{1}{2} \dot{\vec{r}}\, \quat{U}\, \hat{\vec{r}}_0
  \end{equation*}
  \pause
  \begin{align*}
    \frac{d^{2}}{ds^{2}} \quat{U} %
    &= \frac{1}{2} \left( r\, \ddot{\vec{r}} \right)\, \quat{U}\,
    \hat{\vec{r}}_0 + \frac{1}{2} \dot{\vec{r}}\, \left( \frac{1}{2}
      \dot{\vec{r}}\, \quat{U}\, \hat{\vec{r}}_0 \right)\,
    \hat{\vec{r}}_0 \\ %
    &= \frac{1}{2} \left( \ddot{\vec{r}}\, \vec{r} + \frac{1}{2}
      \dot{\vec{r}}^{2} \right) \quat{U} \\ %
    &= \frac{E}{2\mu} \quat{U}
  \end{align*}
\end{frame}

\begin{frame}
  \frametitle{Keplerian orbits}
  \begin{equation*}
    \frac{d^{2}}{ds^{2}} \quat{U} = \frac{E}{2\mu} \quat{U}
  \end{equation*}

  \pause
  \begin{equation*}
    \quat{U} = A\, e^{i\, \omega\, s} + B\, e^{-i\, \omega\, s}
    \qquad
    (i\, \omega)^{2} = \frac{E}{2\mu}
  \end{equation*}
  \pause
  \begin{equation*}
    t = \int \quat{U}\, \co{\quat{U}}\, ds = (A^{2}+B^{2})\, s +
    \frac{A\,B}{\omega}\, \sin (2\,\omega\,s)
  \end{equation*}

  \pause
  \begin{itemize}
   \item Spinor equations are easier \emph{and} better
   \item ``Complex numbers'' contain hidden geometry
  \end{itemize}
\end{frame}

\begin{frame}
  \frametitle{Vector derivative $\nabla$}
  Tensor analysis:
  \begin{equation*}
    \nabla^{a}\, \psi^{b\ldots}
  \end{equation*}

  Differential geometry:
  \begin{equation*}
    d\, \psi
  \end{equation*}
  \pause

  Geometric calculus:
  \begin{equation*}
    \nabla\, \vec{\psi}
    \pause
    = \nabla \cdot \vec{\psi} + \nabla \wedge \vec{\psi}
  \end{equation*}
\end{frame}

\begin{frame}
  \frametitle{Complex analysis}
  \begin{gather*}
    \vec{\nabla}\, \psi = 0 \\ %
    \left( \vec{x} \frac{\partial}{\partial x} + \vec{y}\,
      \frac{\partial}{\partial y} \right) \left( u + v\, \vec{x}\,
      \vec{y} \right) = 0
  \end{gather*}
  \pause
  \begin{gather*}
    \frac{\partial u}{\partial x} = \frac{\partial v}{\partial y}\\
    \frac{\partial u}{\partial y} = -\frac{\partial v}{\partial x}
  \end{gather*}
  \pause

  Cauchy's integral formula:
  \begin{equation*}
    \psi(\vec{a}) = \frac{1}{2\pi i} \oint \frac{\psi(\vec{z})}
    {\vec{z}-\vec{a}}\, ds %
    \pause %
    - \frac{1}{\pi}\int\frac{\vec{\nabla}\psi}{\vec{z}-\vec{a}} i\, dA
  \end{equation*}
\end{frame}

\begin{frame}
  \frametitle{Green's functions (Euclidean case)}
  \begin{gather*}
    G(\vec{x};\vec{y}) = \frac{1}{S_{n}}\, \frac{\vec{x}-\vec{y}} {
      \lvert \vec{x}-\vec{y} \rvert^{n}}
    \\
    \vec{\nabla} G(\vec{x};\vec{y}) = \delta(\vec{x} - \vec{y})
  \end{gather*}

  Cauchy formula in $n$ dimensions:
  \begin{equation*}
    \psi(\vec{y}) = \frac{1} {i\, S_{n}} \oint_{\partial \Omega}
    \frac{\vec{x}-\vec{y}} {\lvert \vec{x}-\vec{y} \rvert^{n}} ds\,
    \psi(\vec{x})
  \end{equation*}
\end{frame}

\begin{frame}
  \frametitle{Stokes' Theorem}
  \begin{itemize}
   \item Fundamental Theorem of Integral Calculus
   \item Kelvin-Stokes Theorem
   \item Green's Theorem
   \item Divergence Theorem
   \item Differential forms:
    \begin{equation*}
      \int_{\partial \Omega} \omega = \int_{\Omega} d\omega
    \end{equation*}
   \item Green's functions
  \end{itemize}
  \pause

  Fundamental theorem of geometric calculus:
  \begin{equation*}
    \int_{\partial \Omega} L(dS) = \int_{\Omega} \mathring{L} (
    \mathring{\nabla} dX)
  \end{equation*}
\end{frame}

\begin{frame}
  \frametitle{Dirac equation}
  Dirac notation:
  \begin{equation*}
    \hat{\gamma}^{\mu} \partial_{\mu}\, \ket{\psi}
    = (m\, + i\, e\, \hat{\gamma}^{\mu}\, A_{\mu}) \ket{\psi}
  \end{equation*}
  \pause

  Geometric calculus:
  \begin{equation*}
    \vec{\nabla} \psi = \left( m\, \psi\, \vec{t} + e\, A\, \psi
    \right)\, \vec{x}\, \vec{y}
  \end{equation*}
  \pause

  Observables:
  \begin{table}
    \begin{tabular}{c c}
      Dirac notation & Geometric calculus \\
      \hline \\[-12pt]
      $\braket{\tilde{\psi} | \psi }
      - I\, \braket{\tilde{\psi} | i \hat{\gamma}_{5} | \psi }$
      & $\psi\, \bar{\psi}$ \\
      $\braket{\tilde{\psi} | \hat{\gamma}_{\mu} | \psi }$
      & $\psi\, \vec{t}\, \bar{\psi}$ \\
      $\braket{\tilde{\psi} | i\, \hat{\gamma}_{\mu\nu} | \psi }$
      & $\psi\, \vec{x}\, \vec{y}\, \bar{\psi}$ \\
      $\braket{\tilde{\psi} | \hat{\gamma}_{\mu}\, \hat{\gamma}_{5}
        | \psi }$
      & $\psi\, \vec{z}\, \bar{\psi}$
    \end{tabular}
  \end{table}
\end{frame}

\begin{frame}
  \frametitle{E\&M}
  Differential forms:
  \begin{equation*}
    dF = 0 \qquad d\ast F = J
  \end{equation*}
  \pause

  Geometric calculus:
  \begin{equation*}
    \vec{\nabla} F = J
  \end{equation*}
\end{frame}

\begin{frame}
  \frametitle{Riemann tensor for Kerr}
  \begin{equation*}
    \mathcal{R}(B) = - \frac{M} {(r-i\, \vec{a} \cdot
      \hat{\vec{r}})^{3}}\, \frac{B - 3\, \vec{r}\, \vec{t}\, B\,
      (\vec{r}\, \vec{t})^{-1}} {2}
  \end{equation*}
\end{frame}

\begin{frame}
  \frametitle{Uses of Geometric Algebra}
  \begingroup
  \fontsize{0.6em}{0.75em}\selectfont
  \begin{columns}
    \begin{column}{0.5\linewidth}
      \begin{itemize}
       \item Complex algebra, complex analysis
       \item Quaternions, Pauli spinors, rotations
       \item Pauli equation
       \item Biquaternions, Dirac spinors, Lorentz group
       \item Relativistic mechanics
       \item Dirac equation
       \item Twistors
       \item Two-spinor calculus
       \item Multiparticle quantum mechanics
       \item Scattering (particles and E\&M waves)
       \item Linear algebra
       \item Tensor algebra
       \item Lie theory
       \item Differential forms
       \item GR
       \item Algebraic geometry
      \end{itemize}
    \end{column}
    \begin{column}{0.5\linewidth}
      \begin{itemize}
       \item Lagrangian and Hamiltonian techniques
       \item Lagrangian field theory
       \item Gauge theory
       \item Projective geometry
       \item Conformal geometry
       \item Conic geometry
       \item Grassmann numbers
       \item Berezin calculus
       \item Green's functions
       \item Stokes' Theorem
       \item Atiyah-Singer index theorem
       \item Combinatorics
       \item Automated theorem proving
       \item Signal processing
       \item Computer vision
       \item Online dating
      \end{itemize}
    \end{column}
  \end{columns}
  \endgroup
\end{frame}

%%%%%%%%%%%%%%%%%%%%%%%%%%%%%%%%%%%%%%%%%%%%%%%%%%%%%%%%%%%%%%%%%%%%%%
%%%%%%%%%%%%%%%%%%%%%%%%%%%%%%%%%%%%%%%%%%%%%%%%%%%%%%%%%%%%%%%%%%%%%%


\part{Summary}

\section{}

\begin{frame}
  \frametitle{Summary}
  \begin{itemize}
   \item GA is easy to understand (and teach)
   \item It unifies and extends familiar math
   \item It makes problems easier
  \end{itemize}

  \vspace{0.4in}

  Why GA is powerful
  \begin{itemize}
   \item Uniform, systematic
   \item Invertible product
   \item Easy access to spinors
   \item Spinor equations are at the heart of physics
   \item Tight integration between algebraic symbology and geometric
    meaning
  \end{itemize}
  This is just the tip of the iceberg...
\end{frame}



%%%%%%%%%%%%%%%%%%%%%%%%%%%%%%%%%%%%%%%%%%%%%%%%%%%%%%%%%%%%%%%%%%%%%%
%%%%%%%%%%%%%%%%%%%%%%%%%%%%%%%%%%%%%%%%%%%%%%%%%%%%%%%%%%%%%%%%%%%%%%

% Empty frame, so I don't accidentally overrun
\part{} \begin{frame} \end{frame}

\part{Extras}

\begin{frame}
  \frametitle{History}
  \begingroup
  \fontsize{0.8em}{1.em}\selectfont
  \begin{itemize}
   \item 1637 Descartes coins ``imaginary'' numbers
   \item 1751 Euler angles
   \item 1843 Hamilton carves quaternion formula into Brougham Bridge
   \item 1844 Grassmann published first edition
   \item 1862 Grassmann publishes heavily rewritten 2nd edition
   \item 1873 Clifford: ``Applications of Grassmann's
    Extensive Algebra''
   \item 1881 Gibbs: ``Elements of vector analysis''
   \item 1987 Hestenes and Sobczyk: ``Clifford algebra to geometric
    calculus''
   \item 2003 Doran and Lasenby: ``Geometric Algebra for Physicists''
  \end{itemize}
  \endgroup
\end{frame}

% \section{Some previous section name}

% \begin{frame}[label=SomeOtherUsefulLabelName]
%   \frametitle{Original slide's name}
%   \begin{center}
%     \hyperlink{SomeUsefulLabelName}%
%     {Ooh!  Extra stuff links back to original}
%   \end{center}
% \end{frame}

\section{}


%%%%%%%%%%%%%%%%%%%%%%%%%%%%%%%%%%%%%%%%%%%%%%%%%%%%%%%%%%%%%%%%%%%%%%
%%%%%%%%%%%%%%%%%%%%%%%%%%%%%%%%%%%%%%%%%%%%%%%%%%%%%%%%%%%%%%%%%%%%%%



\end{document}

%%% Local Variables:
%%% TeX-master: true
%%% End:
